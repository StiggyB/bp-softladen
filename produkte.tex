\chapter{Produkte \& Dienstleistungen}
\section{Produkt}
Softladen.de bietet eine Rundum-Sorglos Lösung für den Einkauf von Software, die Verwaltung von Lizenzen und Installationsdateien, sowie die problemlose Integration in Arbeitsplatzrechner (Windows/Mac/Linux) durch unseren Softwareverteilungs-Client. Attraktive Software-Bundles werden dauerhaft und / oder zu Aktionen zusammengeschnürt und sind dann zu Sonderpreisen zu erwerben.\\

Kleinen und unabhängigen Softwareentwicklern, die keine Möglichkeiten einer umfassenden Vermarktung ihrer Software haben, bieten wir eine Vermarktungsplattform.\\ 

Für ausgewählte Software bieten wir zeitlich befristete Abonnements an. Dadurch können unsere Kunden hohe Anschaffungskosten für spezielle Software reduzieren. Zum Beispiel kostet eine Einzelplatzlizenz für AutoCAD LT\textcopyright 1.450,00 €. In einem Abonnement-Modell kann dieser hohe Preis für eine zeitliche befristete Nutzung deutlich reduziert werden.  \\ 

\section{Zusätzliche Dienstleistung}
Kleine und Mittelständische Unternehmen müssen sich dank unserer Lizensierungsberatung nicht mit unterschiedlichen Lizenzmodellen der unterschiedlichsten Hersteller auseinandersetzen. Anzahl und Gültigkeit der Lizenzen werden automatisch durch den Softwareverteilungs-Client geprüft.\\

Über den Client können Updates für die installierten Programme durchgeführt werden. Dabei werden die von den Herstellern zur Verfügung gestellten Updates der entsprechenden Programme genutzt. Diese Updates können automatisch oder je nach Client-Einstellungen erst nach einer Benachrichtigung, bzw. auf Anfrage des Benutzers, heruntergeladen und installiert werden. So wird eine bequeme Handhabung der Programmaktualität garantiert. Updates müssen nur ausgeführt werden sofern Kompatibilität mit anderen Programmen oder Benutzern benötigt wird. \\

Um Unternehmen und Privatkunden bei den lästigen Lizenzfragen zu helfen, bieten wir darüber hinausgehenden Service wie ein Web-basiertes Tool welches anhand gezielter Fragen ein passendes Lizenzmodell vorschlagen kann, oder auch telefonisch erreichbare Support Mitarbeiter, mit denen diese Fragen im persönlichen Gespräch geklärt werden können.\\ 

Per e-Mail oder Telefon kann ein Kundendienst erreicht werden. Der Kundendienst dient als Ansprechpartner für Probleme bei der Kaufabwicklung. Des Weiteren kann der Kundendienst als Vermittlungsstelle bei Problemen mit der Software selbst genutzt werden und leitet die Kunden an die entsprechenden Kanäle bzw.  Kundendienste der jeweiligen Softwarehersteller weiter. \\

\section{Mögliche zukünftige Entwicklung}
Perspektivisch könnte der aufkommende Markt für Software as a Service (SaaS) eine ins Produkt-Portfolio passende Entwicklung darstellen. Denkbar wäre hier z.B., Software auf virtualisierten Computern in einem Rechenzentrum über eine Desktop-in-Browser Schnittstelle dem Kunden zur Verfügung zu stellen. Somit könnte der Kunde mit sehr wenig Aufwand sogar anspruchsvolle Software auf einfachen Computern verwenden. Neuere Entwicklungen in Virtualisierungstechnologien und Web-Schnittstellen lassen solche Möglichkeiten technisch machbar erscheinen, müssen sich aber erst noch im Betrieb bewähren. Daher erscheint es zu riskant die Unternehmensgründung schon auf dieses Standbein zu stellen.
