\chapter{Marktanalyse}
\section{Gesamtmarkt}
Die Modularität und die damit verbundenen unterschiedlichsten Einsatzbereiche schaffen
für Softladen.de eine große Breite an Marktfeldern.
\begin{itemize}
	\item Große Unternehmen
	\item Mittel bis kleine Unternehmen
	\item Private Kunden
\end{itemize}
Eine Analyse der Marktsegmente Geschäftskunde zeigt, dass es in diesen Bereichen eine große Zahl von Vergleichs- und Konkurrenzanbieter gibt. Außerdem neigen große Unternehmen dazu, Software direkt vom Hersteller zu beschaffen. Eine Markteinführung des Softladen.de in diesen Segmenten unterläge folglich einem stärkeren Konkurrenz- und Preisdruck. Aus dieser Erkenntnis heraus liegt der Geschäftsfokus im Marktsegment Privatkunde. Die Marktanalyse sowie Marketing- und Absatzplanung werden auf diesen Geschäftsbereich ausgerichtet. Weitere Geschäftsfelder, beispielsweise Mittel-bis-kleine Unternehmen, werden in diesem Geschäftsplan als Expansionsmöglichkeiten behandelt.
\section{Marktsegmentierung}
Der Markt für Softladen.de, nämlich der Softwareverkaufsmarkt ist nicht auf Deutschland beschränkt, er ist grundsätzlich weltweit dort einzuordnen, wo Software gebraucht wird. Deutschland hat im Bereich Software eine weltweit bedeutende Rolle und ist daher besonders relevant. Softladen.de soll deshalb zunächst auf Inlandskunden ausgerichtet werden. Im Zuge einer Markteinführung wird der deutschsprachige Raum Europas mit einbezogen. Zur Darstellung einer Marktübersicht Softwareanbieter werten wir Erhebung verschiedener Branchendaten dieses Marktfeldes in der Bundesrepublik Deutschland aus. Wie aus der aktuellen Erhebung der Lünendonk GmbH, Kaufbeuren (Lünendonk\textsuperscript{\textregistered}-Liste 2013\footnote[1]{\url{http://luenendonk.de/wp-content/uploads/2013/05/LUE_Liste_u_PI_2013_Standard_Software_f160520131.pdf})}, hervorgeht, haben die 25 Software-Unternehmen, die im Jahr 2012 die höchsten Umsätze in Deutschland erzielten und jeweils mehr als 60 Prozent ihres Umsatzes im Standard-Software-Geschäft machten, in Deutschland 2012 Inlandsumsätze von zusammen fast 9,4 Mrd. Euro erwirtschaftet. Das entspricht einem inländischen Marktanteil von 55 Prozent (2011: 56%).
Das Leistungsspektrum der Top 25 der Standard-Software-Branche in Deutschland konzentriert sich eindeutig auf die beiden Kernkategorien Standard-Software-Vertrieb (38%) und Software-Wartung (32%). Standard-Software-Einführung und Systemintegration machen im Durchschnitt zusammen gut 10 Prozent aus. Outsourcing, einschließlich Application Service Providing (ASP) und Software as a Service (SaaS), erreicht im Durchschnitt erst 3,8 Prozent der Umsätze. Die restlichen Anteile entfallen auf IT-Beratung (4,5%), Individual-Software-Entwicklung (3,4%), Schulung (3,6%), Hardware-Vertrieb (1,2%) und sonstige Leistungen (3,5%).
Nach Ermittlungen des Brachenverbands Bitkom stieg das Marktvolumen von Standard-AnwendungsSofware, Systemsoftware und Tools 2012 in Deutschland um 5,1 Prozent auf 17,1 Mrd. Euro (2011: 16,2 Mrd. Euro).

Trends
Aktuell entnimmt man der Presse leicht positive Trends für die Gesamtwirtschaftslage, im Besonderen für den Softwarebereich Insgesamt verkauften die vierzehn Standard-Software-Unternehmen aus der Lünendonk\textsuperscript{\textregistered}-Liste, die ihren Hauptsitz bzw. die Mehrheit ihres Grund- oder Stammkapitals in Deutschland haben, für rund 15,3 Mrd. Euro Software-Produkte an Kunden im Ausland. Daraus ergibt sich für 2012 ein Exportanteil am Gesamtumsatz dieser vierzehn Unternehmen (19,5 Mrd. Euro), der mit 78,5 Prozent die entsprechende Quote für das Vorjahr 2011 (76,1%) noch übertrifft. Ein für unser Produkt bedeutendes Ergebnis, ist der Deutscher Software-Markt in 2012 um 4,4 Prozent wächst. Für den Bereich Private Nutzung von Software erwarten Marktkenner ein überdurchschnittliches Wachstum von 20% und mehr. Die Experten berichteten zudem von dem Trend, dass Marktvolumen von SaaS weiter wächst, laut Gartner\footnote[2]{http://www.gartner.com/newsroom/id/1963815}. Beim Markteintritt von Softladen.de wird dieser Trend genutzt, indem den Firmen besondere Angebote und attraktive Preisen angeboten werden. Eine Bewertung der allgemeinen Marktdaten und der Einteilung in einzelne Segmente, führt in unserem Geschäftsbereich zum Hauptsegment Softwarelizenzverkauf und -verwaltung.

\section{Marktpotential}
Für die Quantifizierung des Marktpotentials von Softladen.de als innovativer Softwareanbieter, stehen keine relevanten Daten zur Verfügung. Die Bewertung des Marktpotentials resultiert aus der Erkenntnis, dass es ein kontinuierliches Wachstum der  Softwaremarktvolumen aber immer noch wenige effektive Plattform zur Softwarelizenzverkauf und -verwaltung in Deutschland gibt.

\textbf{Absatzmarkt}
Aus der Analyse des Datenmaterials und Befragungen potentieller Kunden entwickeln wir
Absatzzahlen für die kommenden Jahre. Dabei ist von Bedeutung, dass ein Kunde im
Durchschnitt mehr als ein Software erwerben wird. Bei den uns vorliegenden Befragungsergebnissen
geht es um die Anzahl bis zu 5 Lizenzen verschiedenster Softwares. Für unsere Absatzprognose nehmen wir an, dass jeder Kunde im Durchschnitt 1,5 Softwarelizenz abnehmen
wird.
Jahr 1:
Bestehende Anfragen nach Lieferfähigkeiten des Produktes weiterverfolgen, Vertriebsaufbau, Markteinführung
-> 500 Lizenzen
Jahr 2:
Verkauf starten, mehr Softwarehersteller im Kontakt nehmen, Referenzen zeigen, Marketing und Vertrieb auf das erforderliche Maß ausbauen
Annahme:
Kontaktaufnahme zu ca. 5000 potentiellen Kunden und 100 Softwarehersteller
-> davon ca. 1/3 Interessenten, ca. 1700 Interessenten
-> davon 10 bis 20% Kunden, die „je 1 Lizenz“ abnehmen
-> 340 Lizenzen
Jahr 3:
Vertrieb ausbauen, internationaler Vertrieb, neue Geschäftsfelder aufbauen
Annahme: 
-> Kontaktaufnahme zu ca. 10000 potentiellen Kunden
-> davon ca. 1/3 Interessenten, ca. 3500 Interessenten
-> davon 10% bis 20% Kunden, die „je 2 Lizenzen“ abnehmen
-> 1400 Lizenzen
Jahr 4:
Neben den vorgenannten Maßnahmen den Vertrieb weiter ausbauen, Erweiterung der
Produktfamilie
-> 2500 Lizenzen
Jahr 5:
Neben den vorgenannten Maßnahmen den Vertrieb weiter ausbauen.
-> 5000 Lizenzen

\textbf{Abo-Nutzung}
Neben dem Verkauf der Softwarelizenz Softladen kann das Abonutzungsmodell einen hohen Stellenwert einnehmen, da besonders Kunden an einer kurzzeitigen Nutzung interessiert sind und auf
den Kauf einer permanenten Lizenz des verzichten möchten.
In Gesprächen mit potentiellen Kooperationspartnern werden wir uns der Realisierung dieses Geschäftsfeldes annähern 

\section{Wettbewerber}
Im Hauptmarktsegment „Softwarelizenzverkauf und -verwaltung“ ist erfahrungsgemäß ein relativ hoher Anteil an Stammkunden zu erwarten, die eher bekannte Konzepte bevorzugen. Gleichzeitig sind aber viele Kunden offen gegenüber neuen Ideen. Im Bereich Softwarelizenzverkauf spielen Angebote und Preisestruktur eine sehr wichtige Rolle, d.h. der Kunde sucht nach immer neuen flexiblen Möglichkeiten, um Software kosteneffektiv zu nutzen. Für Softladen.de gibt es im Marktsegment Softwarelizenzverkauf und -verwaltung daher keine Eintrittsbarrieren. Derzeit liegt Wettbewerb nur in Teilbereichen vor. Im Zuge des Markterfolges des Softladen.de ist aber davon auszugehen, dass Wettbewerber in den Markt eintreten werden. Diese werden einerseits Hersteller von Software, andererseits möglicherweise andere Onlineplattformen mit dem Schwerpunkt Softwarelizenzverkauf und -verwaltung  sein.
Als indirekte Wettbewerber bezeichnen wir Softwareretailers, die Boxed Softwares verkaufen.

