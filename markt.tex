\chapter{Marktanalyse}
\section{Gesamtmarkt}
Die Modularität und die damit verbundenen unterschiedlichsten Einsatzbereiche schaffen
für Softladen.de eine große Breite an Marktfeldern.
\begin{itemize}
	\item Große Firmen
	\item Mittel bis kleine Firmen
	\item Private Kunden
\end{itemize}
Eine Analyse der Marktsegmente Geschäftskunde zeigt, dass es in diesen Bereichen eine große Zahl von Vergleichs- und Konkurrenzanbieter gibt. Außerdem gibt es eine sichtbare Wachstumsrate (19% in 2012) von Unternehmen, die für Cloud-Lösungen entschieden haben (Quelle: Gartner). Eine Markteinführung des Softladen.de in diesen Segmenten unterläge folglich einem stärkeren Konkurrenz- und Preisdruck. Aus dieser Erkenntnis heraus liegt der Geschäftsfokus im Marktsegment Privatkunde und Studenten. Die Marktanalyse sowie Marketing- und Absatzplanung werden auf diesen Geschäftsbereich ausgerichtet. Weitere Geschäftsfelder werden in diesem Geschäftsplan als Expansionsmöglichkeiten behandelt.
\section{Marktsegmentierung}
Der Markt für Softladen.de, insbesondere der Softwaremarkt ist nicht auf Deutschland beschränkt, er ist grundsätzlich weltweit dort einzuordnen, wo Software gebraucht wird. Deutschland hat im Bereich Software eine weltweit bedeutende Rolle und ist daher besonders relevant. Im Zuge einer Markteinführung wird der deutschsprachige Raum Europas mit einbezogen. Zur Darstellung einer Marktübersicht Softwareanbieter werten wir Erhebung verschiedener Branchendaten dieses Marktfeldes in der Bundesrepublik Deutschland aus. Sie sind u.a.
den Jahresberichten der…
Trends
Aktuell entnimmt man der Presse leicht positive Trends für die Gesamtwirtschaftslage, im Besonderen für den Softwarebereich (20 Mrd. USDollar für PC Software in 2012 laut DFC Intelligence). Gespräche mit Unternehmen dieses Marktsegmentes bestätigen diese Trends. Ein für unser Produkt bedeutendes Ergebnis, ist der Deutscher Software-Markt in 2012 um 4,4 Prozent wächst. Für den Bereich Private Nutzung von Software erwarten Marktkenner ein überdurchschnittliches Wachstum von 20% und mehr. Die Experten berichteten zudem von dem Trend, dass viele privaten Benutzer den Abo-Nutzungsmodell bevorzugen. Beim Markteintritt von Softladen wird dieser Trend genutzt, indem den Firmen besondere Angebote und attraktive Preisen angeboten werden. Eine Bewertung der allgemeinen Marktdaten und der Einteilung in einzelne Segmente, führt in unserem Geschäftsbereich zum Hauptsegment Softwarelizenzverkauf. Zusätzlich zum konventionellen Softwarelizenzverkauf werden auch verschiedene Möglichkeiten zur Lizenzverwaltung von Software erwartet.

\section{Marktpotential}
Für die Quantifizierung des Marktpotentials von Softladen als innovativer Softwareanbieter, stehen keine relevanten Daten zur Verfügung. Die Bewertung des Marktpotentials resultiert aus verschiedenen Befragungen 
Einschätzung des Marktpotentials
….

Absatzmarkt
Aus der Analyse des Datenmaterials und Befragungen potentieller Kunden entwickeln wir
Absatzzahlen für die kommenden Jahre. Dabei ist von Bedeutung, dass ein Kunde im
Durchschnitt mehr als ein Software erwerben wird. Bei den uns vorliegenden Befragungsergebnissen
geht es um die Anzahl bis zu 5 Lizenzen verschiedenster Softwares. Für unsere Absatzprognose nehmen wir an, dass jeder Kunde im Durchschnitt 1,5 Softwarelizenz abnehmen
wird.
Jahr 1:
Bestehende Anfragen nach Lieferfähigkeiten des Produktes weiterverfolgen, Vertriebsaufbau, Markteinführung
-> 500 Lizenzen
Jahr 2:
Verkauf starten, mehr Softwarehersteller im Kontakt nehmen, Referenzen zeigen, Marketing und Vertrieb auf das erforderliche Maß ausbauen
Annahme:
Kontaktaufnahme zu ca. 5000 potentiellen Kunden und 100 Softwarehersteller
-> davon ca. 1/3 Interessenten, ca. 1700 Interessenten
-> davon 10 bis 20% Kunden, die „je 1 Lizenz“ abnehmen
-> 340 Lizenzen
Jahr 3:
Vertrieb ausbauen, internationaler Vertrieb, neue Geschäftsfelder aufbauen
Annahme: 
-> Kontaktaufnahme zu ca. 10000 potentiellen Kunden
-> davon ca. 1/3 Interessenten, ca. 3500 Interessenten
-> davon 10% bis 20% Kunden, die „je 2 Lizenzen“ abnehmen
-> 1400 Lizenzen
Jahr 4:
Neben den vorgenannten Maßnahmen den Vertrieb weiter ausbauen, Erweiterung der
Produktfamilie
-> 2500 Lizenzen
Jahr 5:
Neben den vorgenannten Maßnahmen den Vertrieb weiter ausbauen.
-> 5000 Lizenzen

Abo-Nutzung
Neben dem Verkauf der Softwarelizenz Softladen kann das Abonutzungsmodell einen hohen Stellenwert einnehmen, da besonders Kunden an einer kurzzeitigen Nutzung interessiert sind und auf
den Kauf einer permanenten Lizenz des verzichten möchten.
In Gesprächen mit potentiellen Kooperationspartnern werden wir uns der Realisierung dieses Geschäftsfeldes annähern 

\section{Wettbewerber}
Im Hauptmarktsegment „Softwareverkauf“ ist erfahrungsgemäß ein relativ hoher Anteil an Stammkunden zu erwarten, die eher bekannte Konzepte bevorzugen. Gleichzeitig sind aber viele Kunden offen gegenüber neuen Ideen. Im Bereich Softwareverkauf spielen Angebote und Preisestruktur eine sehr wichtige Rolle, d.h. der Kunde sucht nach immer neuen flexiblen Möglichkeiten, um Software kosteneffektiv zu nutzen. Für Softladen gibt es im Marktsegment Softwareverkauf daher keine Eintrittsbarrieren. Derzeit liegt Wettbewerb nur in Teilbereichen vor. Im Zuge des Markterfolges des Softladen ist aber davon auszugehen, dass Wettbewerber in den Markt eintreten werden. Diese werden einerseits Hersteller von Software, andererseits Retailers (auch online) mit dem Schwerpunkt Softwarelizenz sein.
Als indirekte Wettbewerber bezeichnen wir Unternehmen..

